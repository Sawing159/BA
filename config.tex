% !TEX root =  master.tex
% 		HYPERREF
%
\usepackage[
	hidelinks=true % keine roten Markierungen bei Links
]{hyperref}

% Zwei eigene Befehle zum Setzen von Autor und Titel. Ausserdem werden die PDF-Informationen richtig gesetzt.
\newcommand{\TitelDerArbeit}[1]{\def\DerTitelDerArbeit{#1}\hypersetup{pdftitle={#1}}}
\newcommand{\AutorDerArbeit}[1]{\def\DerAutorDerArbeit{#1}\hypersetup{pdfauthor={#1}}}
\newcommand{\Firma}[1]{\def\DerNameDerFirma{#1}}
\newcommand{\Kurs}[1]{\def\DieKursbezeichnung{#1}}
\newcommand*{\quelle}[1]{\par\raggedleft\footnotesize Quelle:~#1}


%		FONT AND INPUT ENCODING
%
\usepackage[T1]{fontenc}
\usepackage[utf8]{inputenc}

%		CALCULATIONS
%
\usepackage{calc} % Used for extra space below footsepline

%		LANGUAGE SETTINGS
%
\usepackage[ngerman]{babel} 	% German language
\usepackage[german=quotes]{csquotes} 	% correct quotes using \enquote{}

%\usepackage[english]{babel}   % For english language
%\usepackage{csquotes} 	% Richtiges Setzen der Anführungszeichen mit \enquote{}


%		BIBLIOGRAPHY SETTINGS
%
 \usepackage[backend=biber, autocite=footnote, style=authoryear, dashed=false]{biblatex} 	%Use Author-Year-Cites with footnotes
% \usepackage[backend=biber, autocite=inline, style=ieee]{biblatex} 	% Use IEEE-Style (e.g. [1])
% \usepackage[backend=biber, autocite=inline, style=alphabetic]{biblatex} 	% Use alphabetic style (e.g. [TGK12])

%%%% APA/Harvard-Style (bitte die nächten zwei Zeilen auskommentieren)
%\usepackage[backend=biber, style=apa]{biblatex} 	
%\DeclareLanguageMapping{german}{german-apa}

\DefineBibliographyStrings{ngerman}{  %Change u.a. to et al. (german only!)
	andothers = {{et\,al\adddot}},
}

%%% Uncomment the following lines to support hard URL breaks in bibliography 
%\apptocmd{\UrlBreaks}{\do\f\do\m}{}{}
%\setcounter{biburllcpenalty}{9000}% Kleinbuchstaben
%\setcounter{biburlucpenalty}{9000}% Großbuchstaben


\setlength{\bibparsep}{\parskip}		%add some space between biblatex entries in the bibliography
\addbibresource{bibliography.bib}	%Add file bibliography.bib as biblatex resource


%		FOOTNOTES 
%
% Count footnotes over chapters
\usepackage{chngcntr}
\counterwithout{footnote}{chapter}

%	ACRONYMS
%%%
%%% WICHTIG: Installieren Sie das neueste Acronyms-Paket!!!
%%%
\makeatletter
\usepackage[printonlyused]{acronym}
\@ifpackagelater{acronym}{2015/03/20}
  {%
    \renewcommand*{\aclabelfont}[1]{\textbf{\textsf{\acsfont{#1}}}}
  }%
  {%
  }%
\makeatother

%		LISTINGS
\usepackage{listings}	%Format Listings properly
\renewcommand{\lstlistingname}{Quelltext} 
\renewcommand{\lstlistlistingname}{Quelltextverzeichnis}
\lstset{numbers=left,
	numberstyle=\tiny,
	captionpos=b,
	basicstyle=\ttfamily\small}


%		EXTRA PACKAGES
\usepackage{lipsum}    %Blindtext
\usepackage{graphicx} % use various graphics formats
\usepackage[german]{varioref} 	% nicer references \vref
\usepackage{caption}	%better Captions
\usepackage{booktabs} %nicer Tabs
\usepackage{array}
%\newcolumntype{P}[1]{>{\raggedright\arraybackslash}p{#1}}


%		ALGORITHMS
\usepackage{algorithm}
\usepackage{algpseudocode}
\renewcommand{\listalgorithmname}{Algorithmenverzeichnis }
\floatname{algorithm}{Algorithmus}


%		FONT SELECTION: Entweder Latin Modern oder Times / Helvetica
\usepackage{lmodern} %Latin modern font
%\usepackage{mathptmx}  %Helvetica / Times New Roman fonts (2 lines)
%\usepackage[scaled=.92]{helvet} %Helvetica / Times New Roman fonts (2 lines)

%		PAGE HEADER / FOOTER
%	    Warning: There are some redefinitions throughout the master.tex-file!  DON'T CHANGE THESE REDEFINITIONS!
\RequirePackage[automark,headsepline,footsepline]{scrpage2}
\pagestyle{scrheadings}
\renewcommand*{\pnumfont}{\upshape\sffamily}
\renewcommand*{\headfont}{\upshape\sffamily}
\renewcommand*{\footfont}{\upshape\sffamily}
\renewcommand{\chaptermarkformat}{}

\clearscrheadfoot

\ifoot[\rule{0pt}{\ht\strutbox+\dp\strutbox}DHBW Mannheim]{\rule{0pt}{\ht\strutbox+\dp\strutbox}DHBW Mannheim}
\ofoot[\rule{0pt}{\ht\strutbox+\dp\strutbox}\pagemark]{\rule{0pt}{\ht\strutbox+\dp\strutbox}\pagemark}

\ohead{\headmark}
