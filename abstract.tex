\chapter*{Kurzfassung}
\begingroup
\begin{table}[h!]
\setlength\tabcolsep{0pt}
\begin{tabular}{p{3.7cm}p{11.7cm}}
Titel & \DerTitelDerArbeit \\
Verfasser: & \DerAutorDerArbeit \\
Kurs: & \DieKursbezeichnung \\
Ausbildungsstätte: & \DerNameDerFirma\\
\end{tabular}
\end{table}
\endgroup
Im Rahmen der vorliegenden Bachelorarbeit werden vier verschiedene Tools für Continuous Integration, Continuous Development und Continuous Deployment miteinander verglichen. Dies geschieht im Kontext der Entwicklung eines Angebotes für Continuous Integration, Continuous Development und Continuous Deployment as a Service, welches von der Praxisabteilung erstellt wird. Es soll das beste Tool für den Einsatz bei dem Service evaluiert werden. Die Evaluation der Tools erfolgt dabei durch eine vergleichende Analysemethode. Diese wird der Evaluation vorausgehend aus verschiedenen Methoden ausgewählt und in ihren Abläufen detailliert erklärt. Davor werden zuerst die informationstechnischen Grundlage für die Arbeit geschaffen. Nachdem die theoretischen Grundlagen dieser Arbeit beschrieben wurden, folgt die umfangreiche Analyse nach der zuvor ausgewählten Methodik. Anhand von zuvor mit der Abteilung erarbeiteten Kriterien werden dann die vier Tools evaluiert. Am Ende der Analyse wird eine Reihenfolge der Tools ausgegeben, anhand derer eine Handlungsempfehlung für das weitere Vorgehen gegeben wird. Am Ende wird die Arbeit noch einmal zusammengefasst und kritisch reflektiert. Außerdem wird ein Ausblick auf das weitere Vorgehen nach der Arbeit gegeben und ein Modell für den Service beschrieben. Ziel des Autors ist es, das beste Tool für den Einsatz als Service zu evaluieren. Außerdem soll der Abteilung neue Möglichkeiten zur Automatisierung von Arbeitsschritten während des Entwicklungsprozesses mithilfe neuer Technologien, aufgezeigt und der bestehende Kenntnisstand von Continuous Integration, Continuous Development und Continuous Deployment erweitert werden.


